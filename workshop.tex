\documentclass[11pt, a4paper]{article}
\title{Free and Open Source Software Workshop\\\large\textit{An introduction to tech stuff for lawyers}}
\author{Dewitte, Pierre \and Emanuilov, Ivo \and Biasin, Elisabetta}

\pagestyle{plain}
\usepackage[T1]{fontenc}
\usepackage[style=apa]{biblatex}
\addbibresource{bibliography.bib}
\usepackage[style=british]{csquotes}
\usepackage{microtype}

\begin{document}

\maketitle

\section{Approach for the workshop}

In order for the workshop to be both interesting and instructive, I would suggest to adopt a problem-solution approach. In short, to structure each session around: (i) the description of a technical phenomenon, (ii) the overview of its impact on one's fundamental rights to privacy and data protection, (iii) the presentation of some solutions that can be deployed to mitigate the risks identified and (iv) a list of material for further reading.

\section{Ideas for the sessions}

Space dedicated for some general ideas on the potential sessions.

\begin{itemize}
    \item The functioning of the Domain Name System (DNS)
    \begin{itemize}
        \item Technical overview of the DNS, i.e. the hierarchical and decentralized naming system for computers, services, or other resources connected to the Internet or a private network; overview of the role of the DNS servers in resolving a quiery (from a human-readable domain name to an IP address); a general understanding of the functioning of the Internet Protocol (IP) is a prerequisite. 
        \item Analysis of the privacy implications of using specific DNS servers; explanation of why we should care; link with Scarlet and Breyer case law from the CJEU.
        \item Various solutions can be discussed, from manually changing DNS server to leveraging DNS-based filtering against tracking.
        \item List of publications and resource on the topic.
    \end{itemize}
    \item 
\end{itemize}

\end{document}